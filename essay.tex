\documentclass[14pt]{article}

\usepackage[utf8]{inputenc} 
\usepackage[T1]{fontenc} 
\usepackage{xcolor}
\usepackage[english,ukrainian]{babel}
\usepackage{tempora}
\usepackage{lipsum}
\usepackage{setspace}
\usepackage{geometry}
\usepackage{graphicx}
\graphicspath{ {./images/} }
 \geometry{a4paper, total={170mm,257mm}, left=20mm, top=20mm,}
\pagenumbering{arabic}
\begin{document}
\begin{spacing}{1.175}	
	\begin{titlepage} 
		\newcommand{\HRule}{\rule{\linewidth}{0.3mm}}
		\center 
		
		\textsc{\large Національний технічний університет України
			\\"Київський політехнічний інститут імені Ігоря Сікорського"}\\[1.5cm]
		
		\vspace{5cm}
		\textsc{\large Теорія складності}\\[0.5cm]
		
		\textsc{\large Дослідження \(coNP\)-повної задачі}\\[0.5cm] 
		
		\HRule\\[0.4cm]
		
		{\huge \textcolor{blue}{Задача визначення того, чи даний граф\\
            не має циклу Гамільтона}}\\[0.4cm]
		
		\HRule\\[1.5cm]
		\textsc{\large ФІ-13 Дідух Максим}\\[0.5cm]
		
		\vspace{7.5cm}
		
		\textsc{\large Фізико-технічний інститут}\\[0.5cm]
		\textsc{\large Кафедра математичних методів захисту інформації}\\[0.5cm]
		{\large {2022}} 
	\end{titlepage}
    
    \newpage
    \title{\Large Дослідження \(coNP\)-повної задачі}
    \date{\large 7 грудня 2022}
    \maketitle
    \tableofcontents                                                                            % HOW TO CHANGE THICKNESS here and in numbering????????? 
    \newpage
    \section{\normalfontВступ}
\quadГраф називається \textit{гамільтоновим}, якщо існує цикл, що містить кожну вершину рівно один раз. Такий цикл носить назву гамільтонового.
Поняття пішло від Вільям Гамільтон (англ. \textit{William Hamiltonian}), який вигадав не дуже вдалу гру під назвою "ікосіанська гра" (англ. \textit{"icosian game"}), завданням якої був пошук гамільтонового циклу на додекаедричному графі (і можливо на його підграфах).
\parХоча означення гамільтонового графу дуже схоже з означенням ойлерового графу, виявляється, що ці дві концепції поводяться досить по-різному. Якщо теорема Ойлера дає нам чіткий критерій ойлеровості, то для гамільтонових графів немає аналогічного твердження. Як з'ясувалось, перевірка графу на гамільтовість є \(NP\)-повною задачею.
%%\includegraphics{dodecahedron-modified}
        \subsection{\normalfontПостановка задачі}
        \subsection{\normalfontІсторія виникнення та наявні її модифікації}
    
    \section{\normalfontПрактичне застосування}
    
    \section{\normalfontДоведення складності}

    \section{\normalfontРозв'язок}
        \subsection{\normalfontНаявні методи розв'язку}
        \subsection{\normalfontНаявні ефективні часткові розв'язки задачі та модифікацій}

    \section{\normalfontСписок використаної літератури}

    \newpage
    

    




    
	\end{spacing}
\end{document}
