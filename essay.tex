\documentclass[14pt]{article}

\usepackage[utf8]{inputenc} 
\usepackage[T1]{fontenc} 
\usepackage{xcolor}
\usepackage[english,ukrainian]{babel}
\usepackage{tempora}
\usepackage{lipsum}
\usepackage{setspace}
\usepackage{geometry}
\usepackage{graphicx}
\graphicspath{ {./images/} }
 \geometry{a4paper, total={170mm,257mm}, left=20mm, top=20mm,}
\pagenumbering{arabic}
\begin{document}
\begin{spacing}{1.175}	
	\begin{titlepage} 
		\newcommand{\HRule}{\rule{\linewidth}{0.3mm}}
		\center 
		
		\textsc{\large Національний технічний університет України
			\\"Київський політехнічний інститут імені Ігоря Сікорського"}\\[1.5cm]
		
		\vspace{5cm}
		\textsc{\large Теорія складності}\\[0.5cm]
		
		\textsc{\large Дослідження \(coNP\)-повної задачі}\\[0.5cm] 
		
		\HRule\\[0.4cm]
		
		{\huge \textcolor{blue}{Задача визначення того, чи даний граф\\
            не має циклу Гамільтона}}\\[0.4cm]
		
		\HRule\\[1.5cm]
		\textsc{\large ФІ-13 Дідух Максим}\\[0.5cm]
		
		\vspace{7.5cm}
		
		\textsc{\large Фізико-технічний інститут}\\[0.5cm]
		\textsc{\large Кафедра математичних методів захисту інформації}\\[0.5cm]
		{\large {2022}} 
	\end{titlepage}
    
    \newpage
    \title{\Large Дослідження \(coNP\)-повної задачі}
    \date{\large 7 грудня 2022}
    \maketitle
    \tableofcontents                                                                            % HOW TO CHANGE THICKNESS here and in numbering????????? 
    \newpage
    \section{\normalfontВступ}
        \quadГраф називається \textit{гамільтоновим}, якщо існує цикл, що містить кожну вершину рівно один раз. Такий цикл носить назву гамільтонового.
        Поняття пішло від Вільям Гамільтон (англ. \textit{William Hamiltonian}), який вигадав не дуже вдалу гру під назвою "ікосіанська гра" (англ. \textit{"icosian game"}), завданням якої був пошук гамільтонового циклу на додекаедричному графі (і можливо на його підграфах). \\
        \begin{center}
        \includegraphics{dodecahedron-modified} \\
        \text{мал. 1 \textit{гамільтоновий ціикл на додекаедрі} \textcolor{blue}{\large change resolution}}
        \end{center}
        \parХоча означення гамільтонового графу дуже схоже з означенням ойлерового графу, виявляється, що ці дві концепції поводяться досить по-різному. Якщо теорема Ойлера дає нам чіткий критерій ойлеровості, то для гамільтонових графів немає аналогічного твердження. Як з'ясувалось, перевірка графу на наявність гамільтонового циклу є \(NP\)-повною задачею.

        \newcommand{\nonhamcycle}{\textbf{NONHAMCYCLE }}
        \newcommand{\hamcycle}{\textbf{HAMCYCLE }}
        \subsection{\normalfontПостановка задачі}
        \quad Дано: неорієнтований граф \(G = (V,E)\), де \(V = \{v_1, v_2, \dots v_n\}\) - множина вершин, \(E\ = \{e_1, e_2, \dots v_k\}\) - множина ребер. Перевірити, що даний граф не має циклу гамільтона. Цю задачу будемо позначати \nonhamcycle .    
        \subsection{\normalfontІсторія виникнення та наявні її модифікації}
    
    \section{\normalfontПрактичне застосування}
    
    \section{\normalfontДоведення складності}
    \quad Доведемо, що \nonhamcycle є \(coNP\)-повною. Для цього нам знадобиться довести додаткове твердження. \\\\
    \textbf{Твердження 1:} задача перевірки графа на наявність гамільтонового циклу належить класу \(NP\)-повних задач (\textit{далі:} \hamcycle)\\
    \rule{0.7em}{0.7em}\\
    Доведення цього факту складається з двох кроків:
    
    \\ 
    \quad 1. \hamcycle \(\epsilon\) \(NP\)?\\
        Якщо довільна задача, належить класу \(NP\), тоді, маючи «сертифікат», який є розв'язком зієї задачі та екземпляр проблеми (граф \(G\) і додатне ціле \(k\), у цьому випадку), ми зможемо верифікувати ( перевірити, чи надане рішення правильне чи ні) сертифікат за поліноміальний час. Сертифікат — це послідовність вершин, що утворюють гамільтонів цикл у графі. Ми можемо верифікувати розв'язок, перевіривши, що всі вершини належать графу, і, що кожна пара вершин, що належать розв’язку — суміжна.\\ Це можна зробити за поліноміальний час, тобто \(O(V + E)\), ось
        псевдокод верифікації сертифікату для графу \(G(V, E)\):
    
        \\
        \quad \texttt{value = \(1\)}
        
        \\
        \quad \texttt{для кожної пари \( \{u, v\}\) у підмножині \(V`\):}
        
        \\
        \quad \texttt{\quad перевірити, що між цими вершинами є ребро}
        
        \\
        \quad \texttt{\quad якщо ребра немає, то повернути \(0\) і завершити цикл}
        
        \\
        \quad \texttt{якщо value = \(1\):}
        
        \\
        \quad \texttt{\quad то повернути \(1\) (розв'язок коректний)}
        
        \\
        \quad \texttt{інакше:}
        
        \\
        \quad \texttt{\quad повернути \(0\) (розв'язок неправильний)}\\
        
        \\
        
        \\
        \quad 2. \hamcycle \(NP\) складна ? 
        
        \\
        
    %\end{enumerate}

    \hspace{15cm} \rule{0.7em}{0.7em}
    
                                            %\item \nonhamcycle є \(coNP\)-повною


    
    
    
    
        
        

    \section{\normalfontРозв'язок}
        \subsection{\normalfontНаявні методи розв'язку}
        \subsection{\normalfontНаявні ефективні часткові розв'язки задачі та модифікацій}

    \section{\normalfontСписок використаної літератури}

    \newpage
    

    




    
	\end{spacing}
\end{document}
