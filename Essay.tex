\documentclass[14pt]{article}
\usepackage[utf8]{inputenc} 
\usepackage[T1]{fontenc} 
\usepackage{xcolor}
\usepackage[english,ukrainian]{babel}
\usepackage{tempora}
\usepackage{lipsum}
\usepackage{setspace}
\usepackage{geometry}
\usepackage{graphicx}
\usepackage{amsmath}
\usepackage{tikz}
\usepackage{listings}
\usetikzlibrary{calc,arrows.meta,positioning}
\graphicspath{ {./images/} }
\geometry{a4paper, total={170mm,257mm}, left=20mm, top=20mm,}
\pagenumbering{arabic}
\usepackage{fancyhdr}
\pagestyle{fancy}
\fancyhf{}
\fancyhead[R]{\thepage}
\fancyheadoffset[L]{\sectionwidth}
\fancyhead[L]{\expandafter\textit{Теорія складності}}


\begin{document}
\begin{spacing}{1.175}	
	\begin{titlepage} 
		\newcommand{\HRule}{\rule{\linewidth}{0.3mm}}
		\center 
		
		\textsc{\large Національний технічний університет України
			\\"Київський політехнічний інститут імені Ігоря Сікорського"}\\[1.5cm]
		
		\vspace{5cm}
		\textsc{\large Теорія складності}\\[0.5cm]
		
		\textsc{\large Дослідження \(coNP\)\textit{- повної} задачі}\\[0.5cm] 
		
		\HRule\\[0.4cm]
		
		{\huge \text{Задача негамільтонового циклу}}\\[0.4cm]
		
		\HRule\\[1.5cm]
		\textsc{\large ФІ-13 Дідух Максим}\\[0.5cm]
		
		\vspace{9cm}
		
		\textsc{\large Фізико-технічний інститут}\\[0.5cm]
		\textsc{\large Кафедра математичних методів захисту інформації}\\[0.5cm]
		{\large {2022}} 
	\end{titlepage}
    
    
    
    \newpage
    \title{\Large Дослідження \(coNP\)\textit{- повної} задачі}
    \date{\large 7 грудня 2022}
    \maketitle
    \tableofcontents                                                                            %CHANGE THICKNESS HERE AND IN NUMBERING))))0)
    \newpage
    \section{\normalfontВступ}
        \quadГраф \(G\) називається \textit{гамільтоновим}, якщо існує цикл, що містить кожну вершину рівно один раз. Такий цикл носить назву гамільтонового. Також слід зазначити, що існує поняття гамільтоновго шляху (це знадобиться при доведенні складності). Граф \(G\) має \textit{гамільтонів шлях} з вершини \(s\) у вершину \(t\), якщо можна прокласти марштрут між цими вершинами, який містить усі вершини рівно один раз.
        
        \\
        \quad Взагалі, поняття пішло від Вільям Гамільтон (англ. \textit{William Hamiltonian}), який вигадав не дуже вдалу гру під назвою "ікосіанська гра"
        \hspace{0.08cm}(англ. \textit{"icosian game"}), завданням якої був пошук гамільтонового циклу на додекаедричному графі (і можливо на його підграфах). Проблема негамільтонового циклу — класична задача теорії графів із широким спектром застосувань. Мета задачі полягає в тому, щоб визначити, чи містить заданий граф негамільтонів цикл (цикл, у якому відвідуються не всі вершини). Ця проблема була широко досліджена в літературі, але в багатьох випадках залишається відкритою \\
        \begin{center}
        \tikzset{unode/.style = {
                circle, 
                draw=gray!40!black, 
                thick,
                fill=gray!40!,
                inner sep=2.3pt,
                minimum size=2.3pt } }
        \tikzset{uedge/.style = {
            draw=cyan!20!black, 
            very thick} }

        \begin{tikzpicture}[scale=1.6 ]
            \foreach \x in {0,1,2,3,4}{
                \node[unode] (o\x) at (18+\x*72:2cm) {};
                \node[unode] (i\x) at (18+\x*72:1.4cm) {};
                \node[unode] (ii\x) at (54+\x*72:1cm) {};
                \node[unode] (iii\x) at (54+\x*72:0.5cm) {};
            }
            \foreach \x in {0,1,2,3,4}{
                \path[uedge] (o\x) edge (i\x); 
                \path[uedge] (ii\x) edge (iii\x);
            }
            \path[uedge][orange] (o0)--(o1)--(o2)--(o3)--(o4)--(o0);
            \path[uedge][orange] (o3)--(i3);
            \path[uedge] (o3)--(o4);
            \path[uedge][orange] (o4)--(i4);
            \path[uedge][orange] (iii0)--(iii1)--(iii2)--(iii3)--(iii4)--(iii0);
            \path[uedge] (iii3)--(iii4);
            \path[uedge][orange] (ii4)--(iii4);
            \path[uedge][orange] (ii3)--(iii3);
            \path[uedge][orange] (i0)--(ii0)--(i1)--(ii1)--(i2)--(ii2)--(i3)--(ii3)--(i4)--(ii4)--(i0);
            \path[uedge] (i3)--(ii3);
            \path[uedge] (i4)--(ii4);

        \end{tikzpicture} 
        \\
        \text{мал. 1 \textit{гамільтоновий цикл на додекаедрі}}
        \end{center}
        
        \parХоча означення гамільтонового графу дуже схоже з означенням ойлерового графу, виявляється, що ці дві концепції поводяться досить по-різному. Якщо теорема Ойлера дає нам чіткий критерій ойлеровості, то для гамільтонових графів немає аналогічного твердження. Як з'ясувалось, перевірка графу на наявність гамільтонового циклу є \(NP\)\textit{- повною} задачею.

        \newcommand{\nonhamcycle}{\textit{NON-HAM-CYCLE}}
        \newcommand{\hamcycle}{\textit{HAM-CYCLE}}
        \newcommand{\dhampath}{\textit{D-HAM-PATH}}
        \newcommand{\tsat}{\textit{3SAT}}
        \newcommand{\bolddot}{\textbf{.}}

        
        \subsection{\normalfontПостановка задачі}
        \quad Дано: неорієнтований граф \(G = (V,\,E)\), де \(V = \{v_1, v_2, \dots v_n\}\) - множина вершин, \(\,E\ = \{e_1, e_2, \dots e_m\}\) - множина ребер. Перевірити, що даний граф не має циклу гамільтона. Цю задачу будемо позначати \nonhamcycle .    
        \subsection{\normalfont Історичний екскурс}
        
        Задача негамільтонового циклу (\nonhamcycle) — це математична задача, яка існує з кінця 1960-х років, що передбачає перевірку графа на не наявність циклу гамільтона, але наявність усіх інших циклів. Проблема була вперше поставлена в 1969 році Вільямом Тутте і Річардом Тістлетвейтом і стала областю активних досліджень як у математиці, так і в інформатиці.

        \nonhamcycle\hspace{0.05cm} можна розглядати як узагальнення класичної проблеми гамільтонового циклу. У цій узагальненій версії ми хочемо знайти цикли на довільному графі, які не є гамільтоновими, але допускаються всі інші цикли довжиною не менше трьох. \nonhamcycle\hspace{0.05cm} має низку застосувань у галузі теорії графів і суміжних областях, таких як оптимізація мережі, планування та проектування схем. Крім того, проблема має значення для вивчення орієнтованих графів, оскільки вона служить схемою для створення негамільтонових циклів у орієнтованих графах.

        Дивно, але незважаючи на те, що \hspace{0.05cm}\nonhamcycle\hspace{0.05cm} було запропоновано понад 50 років тому, незначний прогрес був досягнутий у пошуку алгоритму поліноміального часу. Відомо, що проблема \(NP\)\textit{- складна}; проте було розроблено низку евристичних та апроксимаційних алгоритмів, у тому числі на основі локального пошуку, ланцюга Маркова Монте-Карло та генетичних алгоритмів.

        Задача негамільтонового циклу все ще залишається головним відкритим питанням у теоретичній інформатиці та математиці. Хоча за останні роки було досягнуто значного прогресу, все ще потрібно багато працювати, щоб визначити, чи можна розв’язати проблему за поліноміальний час. Тим часом дослідження цієї захоплюючої проблеми продовжують надихати математиків та інженерів із різних галузей.
    
    \section{\normalfontПрактичне застосування}
    Негамільтонові цикли мають широкий спектр застосування, оцінити вплив цього поняття та теорії, що побудувалась навколо нього неможливо, тому я спробую навести лише деякі (очевидні та не дуже) приклади застосування з окремих галузей математики, інформатики та не тільки:
    \begin{itemize}
       
        \item Мережі
        \begin{itemize}
            
            \item Проектування мереж: негамільтонові цикли можна використовувати для проектування надійних мереж, у яких існує кілька надлишкових шляхів між двома точками.
        
            \item Проектування мережі: негамільтонові цикли часто використовуються при проектуванні мереж, де метою є створення шляху, який відвідує всі вузли мережі, не проходячи через жодний вузол більше одного разу.
        
            \item Маршрутизація: негамільтонові цикли можна використовувати для маршрутизації інформації через мережу комп’ютерів. Це особливо актуально для комп’ютерних мереж, які містять вузли з різною потужністю або функціями. У цьому випадку можна використовувати негамільтонів цикл, щоб знайти шлях через мережу, який використовує переваги окремих вузлів, мінімізуючи затримку та перевантаження.
        
            \item Маршрутизація: негамільтонові цикли зазвичай використовуються в алгоритмах мережевої маршрутизації, таких як алгоритм Беллмана-Форда, який знаходить найкоротший шлях між двома вузлами в мережі, спочатку будуючи негамільтонів цикл.
            
            \item Проектування мереж: негамільтонові цикли можна використовувати для створення ефективних мереж, таких як комунікаційні мережі, транспортні мережі та ланцюги поставок. Використовуючи підхід негамільтонового циклу, ці мережі можуть бути більш ефективними, ніж традиційні.
            
        \end{itemize}


        \item Аналіз даних
        \begin{itemize}
            
            \item Аналіз даних: негамільтонові цикли можна використовувати для виявлення закономірностей у наборах даних (допомагає визначити зв’язки між різними частинами даних).
            
            \item Машинне навчання: негамільтонові цикли також використовуються в алгоритмах машинного навчання, таких як нейронні мережі та опорні векторні машини, для визначення закономірностей і прогнозування.
                
            \item Візуалізація даних: негамільтонові цикли корисні для візуалізації даних у формі графіків, наприклад, коли використовується силово-спрямований алгоритм компонування графа \textit{(force-directed graph layout algorithm)}. Такі алгоритми, використовують концепцію сил «відштовхування» та «притягання» для розташування вузлів у межах графа. Гамільтонів цикл не обов’язково є ідеальним для такого виду візуалізації, оскільки його ребра можуть стати занадто стиснутими, якщо є лише кілька вузлів або якщо вузли розташовані дуже близько один до одного. Негамільтонові цикли забезпечують більшу гнучкість і краще підходять для відображення складних даних у візуально цікавий та інформативний спосіб.

            \item Розпізнавання шаблонів: негамільтонові цикли можна використовувати для пошуку шаблонів або трендів у великих наборах даних.
        
        \end{itemize}


        \item Криптографія
        \begin{itemize}
             
             \item Негамільтонові цикли також використовуються в певних криптографічних системах, таких як обмін ключами Діффі-Хеллмана та криптографія еліптичної кривої, для безпечної передачі даних через Інтернет.
        
        \end{itemize}

        \newpage
        \item Планування та логістика
        \begin{itemize}
            
            \item Планування: негамільтонові цикли можна використовувати для планування завдань у розподілених системах, оскільки вони забезпечують ефективний спосіб мінімізації загального часу, необхідного для виконання всіх завдань.

            \item Планування: негамільтонові цикли можна використовувати для планування завдань на паралельному процесорі, наприклад планування завдань на роботі-мануалі на виробничому підприємстві.
            
            \item Планування: негамільтонові цикли можна використовувати для створення оптимальних розкладів для складних завдань, таких як планування маршрутів польоту або оптимізація операцій ланцюга поставок.
            
            \item Оптимізація логістики: негамільтонові цикли можна використовувати для вирішення проблеми комівояжера в оптимізації логістики, де метою є знайти найкоротший маршрут для транспортного засобу доставки серед кількох зупинок.

            \item Планування шляху: негамільтонові цикли також можна використовувати для планування шляху, особливо коли на шляху є перешкоди. Цикл можна використовувати для уникнення цих об’єктів під час пошуку найкоротшого маршруту між двома точками.

            \item Маршрутизація транспортного засобу: негамільтонові цикли можна використовувати в задачах маршрутизації транспортного засобу, де метою є знайти найефективніший маршрут для руху транспортного засобу між набором місць.
        
        \end{itemize}

            
        \item Обробка зображень
        \begin{itemize}
    
            \item Обробка зображень: алгоритми на основі графів з негамільтоновими циклами використовуються при сегментуванні зображень на області чи об’єкти.
            
            \item Обробка зображень: негамільтонові цикли також використовуються в задачах обробки зображень, таких як виявлення країв на зображеннях і визначення форм на цифрових фотографіях.

            \item Мистецтво та дизайн: негамільтонові цикли можна використовувати для створення естетично привабливих візерунків у мистецтві та дизайні.
        
        \end{itemize}
        
        \item Робототехніка
        
        \begin{itemize}
            \item Роботна навігація: негамільтонові цикли створюють основу для досліджень певного середовища роботами, одночасно уникаючи перешкод.

            \item Робототехніка: роботи можуть використовувати негамільтонові шляхи для пересування з місця на місце, уникаючи перешкод і досліджуючи навколишнє середовище.
        
        \end{itemize}


        \item Розфарбування графів
        \begin{itemize}
            
            \item негамільтонові цикли можна використовувати в задачах розфарбовування, щоб розділити граф на два або більше менших підграфів з оптимізованим балансом кольорів.
            
            \item негамільтонові цикли можна використовувати в алгоритмах розфарбовування графів, де метою є розфарбувати всі вершини графа без будь-яких двох суміжних вершин, які мають однаковий колір.
        
        \end{itemize}


        \item Інше
        \begin{itemize}
            
            \item Промислові процеси: негамільтонові цикли використовуються в промислових процесах, таких як хімічне машинобудування та автоматизація. Наприклад, їх можна використовувати для оптимізації маршрутів постачання матеріалів і виробництва на заводах.

            \item Секвенування ДНК: негамільтонові цикли використовуються в алгоритмах секвенування ДНК. Алгоритми використовують підхід негамільтонового циклу для аналізу відносних положень нуклеотидів у молекулі ДНК.

            \item Проблеми оптимізації: негамільтонові цикли також можна використовувати для вирішення проблем оптимізації, таких як проблема комівояжера. На відміну від рішення, яке ґрунтується на гамільтоновому циклі, який гарантовано знаходить оптимальне рішення, але є дорогим з точки зору обчислень, негамільтонів цикл часто можна використовувати для знаходження майже оптимального рішення за менший час.
        
        \end{itemize}
    \end{itemize}
    
    \section{\normalfontДоведення складності}

    Доведемо, що \nonhamcycle\hspace{0.05cm} є \(coNP\)\textit{- повною}. Для цього нам знадобиться довести додаткове твердження. \\\\
    \textbf{Твердження 1:} задача перевірки графа на наявність гамільтонового циклу належить класу \(NP\)\textit{- повних} задач, (\textit{далі:} \hamcycle).
    
    \(\triangleright\) Спочатку покажемо, що задача перевірки орієнтованого графу на наявність гамільтоновго шляху з вершини \(s\) у вершину \(t\) є \(NP\)\textit{- повною}, (далі: \dhampath).

    1. \dhampath\hspace{0.05cm} належить класу \(NP\)\textit{- повних} задач?\\
    Нехай \(G\) — орієнтований граф. Ми можемо перевірити чи потенційний шлях \(s \to \dots \to t\) є гамільтоновим за поліноміальний час. Тепер спробуємо побудувати поліноміальне зведення \tsat \,\(\le_p\) \dhampath, щоб закінчити доведення повноти.
    
    Нехай кон'юнктивна нормальна форма \(\phi = \bigwedge\limits_{i=1}^m \phi_{i}\) має \(n\) змінних і \(m\) кон'юнкцій. Щоб спростити доведення, припустимо, що жодна кон'юнкція в \(\phi\) не містить змінної \(x_i\) і її заперечення \(\bar{x_i}\).  
    
    \begin{enumerate}
        \item  Спочатку, для кожної змінної \(x_i\), ми генеруємо \(2m+1\) вершину з назвою \(v_{i,j}\) і додаємо орієнтовані ребра \((v_{i,j}\), \(v_{i,j+1})\) і \((v_{i,j+1}\), \(v_{i,j})\), для \(0 \le j \le 2m\).
        
        \item  Далі ми з’єднуємо вершини, пов’язані з різними змінними, додаючи чотири спрямовані ребра \((v_{i,0}\), \(v_{i+1,0})\), \((v_{i,0}\), \(v_{i+1,2m+1})\), \((v_{i,2m+1}\), \(v_{i+1,0})\) та \((v_{i,2m+1}\), \(v_{i+1,2m+1})\).
        
        \item  Тепер створюємо вершини під назвою \(c_{j}\). Якщо \(x_i\) з'являється у кон'юнкції \(\phi_{j}\) без доповнення, додаємо орієнтовані ребра
        \((v_{i,2j-1}\), \(c_{j})\) та \((c_{j}\), \(v_{i,2j})\). Інакше, якщо \(\,x_i\,\) міститься з доповненням, то додаємо орієнтовані ребра \((v_{i,2j}\), \(c_{j})\) та \((c_{j}\), \(v_{i,2j-1})\).
        
        \item В кінці, ми додаємо дві додаткові вершини \(s\) і \(t\). Після цього додаємо нові ребра \((s\), \(v_{1,1})\), \((s\), \(v_{1,2m+1})\), \\\((v_{n,1}\), \(t)\) та \((v_{n,2m+1}\), \(t)\). Граф \(G_{\phi}\) згенерований кон'юнктивною нормальною формою \(\phi = \bigwedge\limits_{i=1}^m \phi_{i}\) (\textit{див. рис.2}).
    \end{enumerate}
    \\

    \quad Тепер, треба показати, що \(\phi\) задовільна тоді та лише тоді, коли у графі \(G_{\phi}\) є гамільтонів шлях з \(s\) у \(t\). Припустимо, що \(\phi\) - задовільна. Тоді ми можемо відвідати кожну вершину починаючи з \(s\) йдучи з \(v_{i,0}\) до \(v_{i,2m+1}\) з ліва на право, якщо \(x_i\) - позитивна, і з \(v_{i,2m+1}\) до \(v_{i,0}\) з права на ліво, якщо \(x_i\) - негативна, і закінчити в \(t\) після відвідування \(v_{n,0}\) або \(v_{n,2m+1}\). Крім того, кожна кон'юнктивна вершина може бути відвідана, згідно з припущенням, що кожна кон'юнкція \(\phi_{i}\) має декілька літерлів \(l_i = x_k\) або \(l_i = \bar{x_i}\), які є позитивними. Кожна вершина \(c_j\) може бути відвідана використовуючи ребра \((x_{k,2j-1}\), \(c_{j})\) або \((c_{j}\), \(x_{k,2j})\) у першому випадку, і \((x_{k,2j}\), \(c_{j})\) та \((c_{j}\), \(x_{k,2j-1})\) у другому випадку, оскільки шлях йде вправо на графі для позитивних змінних, і вліво для негативних змінних. Тому \(G_{\phi}\) має гамільтонів шлях, якщо \(\phi\) задовільна.
    
    \\
    \quad І навпаки, припустимо, що \(G_{\phi}\) має гамільтонів шлях (\(s, t\)), позначимо його \(P\). Зазначимо, що \(G_{\phi}\) без кон'юнктивних вершин є гамільтоновим, тому нам потрібно показати, що шлях не "зламається" \hspace{0.08cm}після відвідування кон'юнктивної вершини. Якщо більш формальніше: потрібно показати, що якщо \(P\) відвідує \(v_{i, 2j-1}\), \(c_j\), \(v\) у відповідному порядку, тоді \(v = v_{i,2i}\). Нехай \(v \neq v_{i,2i}\). Тоді зазначимо, що єдиними вершинами, які входять в \(v_{i,2j}\) є \(v_{i,2j-1}\), \(c_j\), \(v_{i,2j+1}\), а виходять з нього лише \(v_{i,2j-1}\) та \(v_{i,2j+1}\). Тому \(v_{i,2j}\) має бути відвідана з \(v_{i,2j+1}\), але тепер шлях зламається і не зможе продовжуватись, оскільки \(v_{i,2j-1}\) вже відвідано. Аналогічним чином можна показати, що \(P\) відвідує \(v_{i, 2j}\), \(c_j\), \(v\) у відповідному порядку, тому \(v = v_{i,2j-1}\). Отже, гамільтонів шлях в \(G_{\phi}\) відвідує вершини в порядку від \(x_1\) до \(x_n\), чергуючи кон'юнктивні вершини між змінними , пов'язаними з тією самою змінною. Тому, значення змінних \(x_i\) добре визначається, якщо помітити, у якому напрямку шлях йде через вершини \(x_i\) у графі \(G_{\phi}\). За побудовою, це призначення задовольнить \(\phi\), оскільки призначення робить літерал у кожному пункті істинним.
    
    \quad Зведення відбувається за \(\mathcal{O}(mn)\) час, який є поліномом довжини вхідних даних.
    \\


    \\
    \quad 2. \hamcycle\hspace{0.05cm} належить класу \(NP\,\)?\\
        Якщо довільна задача, належить класу \(NP\), тоді, маючи «сертифікат», який є розв'язком цієї задачі та екземпляр проблеми (граф \(G\) і додатне ціле \(k\), у цьому випадку), ми зможемо верифікувати (перевірити, чи надане рішення правильне чи ні) сертифікат за поліноміальний час. Сертифікат — це послідовність вершин, що утворюють гамільтонів цикл у графі. Ми можемо верифікувати розв'язок, перевіривши, що всі вершини належать графу,\hspace{0,05cm} і, що кожна пара вершин, що належать розв’язку — суміжна.\\ Це можна зробити за \(O(V + E)\), тобто поліноміальний час, ось псевдокод верифікації сертифікату для графу \(G(V, E)\):
    
        \quad \texttt{bool value = \(1\)}
        
        \quad \texttt{для кожної пари \( \{u, v\}\) у підмножині \(V`\):}
        
        \quad \texttt{\quad перевірити, що між цими вершинами є ребро}
        
        \quad \texttt{\quad якщо ребра немає, то повернути \(0\) і завершити цикл}
        
        \quad \texttt{якщо value = \(1\):}
        
        \quad \texttt{\quad то повернути \(1\) (розв'язок коректний)}
        
        \quad \texttt{інакше:}
        
        \quad \texttt{\quad повернути \(0\) (розв'язок некоректний)}\\

        
        \\
        \quad 3. \hamcycle\hspace{0.05cm} належить класу \(NP\)\textit{-складних }задач? \\
        Щоб довести, що \hamcycle\hspace{0.05cm} є \(NP\)\textit{-складною}, ми повинні звести вже відому \(NP\)\textit{-складну} задачу до даної. Побудуємо зведення від \dhampath\hspace{0.05cm} до \hamcycle. Кожен екземпляр задачі \dhampath\hspace{0.05cm} містить граф \(G = (V,\,E) \), який може бути конвертований в задачу \hamcycle, яка містить граф \(G' = (V',\,E')\). Граф \(G'\) будемо будувати таким чином:
        \begin{enumerate}
            \item \(V' = V \cup \{v_{new}\}\), де \(v_{new}\) - це додаткова вершина, яка поєднана ребрами з усіма вершинами графу \(G\). \(E_{new} = \{\,(v_{new},\,v_i)\,|\,\,\forall v_i\,\in\,V\}\) - множина усіх ребер, що поєднують вершину \(v_{new}\) з графом \(G\).
            \item \(E'\ = E \cup E_{new}\) 
            \item Припустимо, що граф \(G\) містить гамільтонів шлях \(P\), який починається у випадковій вершині, позначимо її як \(v_{start}\) та закінчується у вершині \(v_{end}\). Тепер, оскільки ми поєднали усі вершини з множини \(V\) з вершиною \(v_{new}\), ми можемо розширити гамільтонів шлях \(P\) о гамільтонового циклу, використовуючи ребра \((v_{end},\, v_{new})\) та \((v_{new},\, v_{start})\) відповідно. Тепер маємо граф \(G'\), який містить цикл, що проходить усі вершини рівно один раз.
            \item Ми припускаємо, що граф \(G'\) містить гамільтонів цикл, який проходить через усі вершини, включаючи \(v_{new}\). Тепер, щоб перетворити цей цикл на гамільтонів шлях, ми видаляємо усі ребра (у циклі), що містять вершину \(v_{new}\). Отриманий шлях буде покривати усі вершини з множини \(V\) і будуть робити це рівно один раз.
        \end{enumerate}
        
        \quad Тому ми можемо стверджувати, що граф \(G'\) містить гамільтонів цикл, якщо граф \(G\) містить гамільтонів шлях. Отже, довільний екземпляр задачі \hamcycle\hspace{0,05cm} зводить до екземпляру задачі \dhampath. З чого можна зробити висновок, що \hamcycle\hspace{0,05cm} є \(NP\)\textit{- складною}.\\
        
        \quad Підсумовуючи все вище сказане, маємо, що \hamcycle належить класу \(NP\) і є \(NP\)-складною, тому можна зробити висновок, що \hamcycle є \(NP\)\textit{- повною}.

        
    \hspace{15cm} \rule{0.7em}{0.7em}
    
    \\
    \quad Тепер, довівши \textit{Твердження 1}, можемо виконати фінальні викладки. Маємо, що \hamcycle\hspace{0,05cm} належить класу \(NP\) та \nonhamcycle\hspace{0,05cm} є доповненням задачі \hamcycle, отже задача \nonhamcycle\hspace{0,05cm} належить класу \(coNP\). Також, врахувавши, що \hamcycle\hspace{0,05cm} є \(NP\)\textit{- повною}, можна зробити висновок, що \nonhamcycle\hspace{0,05cm} є  \(coNP\)\textit{- повною}.
        
        \begin{center}
        \tikzset{
            circle node/.style={
            circle,
            draw,
            fill=white,
            minimum size=1.3cm
            }
        }
        \begin{tikzpicture}
        \begin{scope}[every node/.style = {circle, thick, draw, minimum size = 2.7em, fill = gray!40}, edge_style/.style={draw, ultra thick}]
            
            \node (s) at (-9,0) {s};
            %row 1
            \node (v_{1,0}) at (-15,-3) {\(v_{1,0  }\)};
            \node (v_{1,1}) at (-12.5,-3) {\(v_{1,1}\)};
            \node (v_{1,2}) at (-10,-3) {\(v_{1,2}\)};
            \node[draw = white, fill = white] (d1) at (-7.5,-3) {. . . };
            \node (v1) at (-5, -3) {};
            \node (v_{1,2m+1}) at (-2.5,-3) {\(v_{1,2m+1}\)};
           
            %row 2
            \node (v_{2,0}) at (-15, -7) {\(v_{2,0}\)};
            \node (v_{2,1}) at (-12.5,-7) {\(v_{2,1}\)};
            \node (v_{2,2}) at (-10,-7) {\(v_{2,2}\)};
            \node[draw = white, fill = white] (d2) at (-7.5,-7) {. . . };
            \node (v2) at (-5, -7) {};
            \node (v_{2,2m+1}) at (-2.5,-7) {\(v_{2,2m+1}\)};
           
            %row 3
            \node (v_{3,0}) at (-15,-10) {\(v_{3,0}\)};
            \node[draw = white, fill = white] (d3) at (-6.5,-10) {\dots};
            \node[draw = white, fill = white] (d3) at (-11,-10) {\dots};
            \node (v_{3,2m+1}) at (-2.5,-10) {\(v_{2,3m+1}\)};            
            
            %row n
            \node (v_{n,0}) at (-15,-21) {\(v_{n,0}\)};
            \node (v_{n,2m+1}) at (-2.5,-21) {\(v_{n,2m+1}\)};  
            \node[draw = white, fill = white] (d3) at (-6.5,-21) {\dots};
            \node[draw = white, fill = white] (d3) at (-11,-21) {\dots};
            
            %6 addition
            \node (v_{i,2m-1}) at (-8,-13) {\(v_{i,2m-1}\)};
            \node (v_{i,2m}) at (-5,-13) {\(v_{i,2m}\)};
            
            \node (v_{j,2m-1}) at (-8,-16) {\(v_{j,2m-1}\)};
            \node (v_{j,2m}) at (-5,-16) {\(v_{j,2m}\)};

            \node (v_{k,2m-1}) at (-8,-19) {\(v_{k,2m-1}\)};
            \node (v_{k,2m}) at (-5,-19) {\(v_{k,2m}\)};


            \node (c_1) at (1, -11) {\(c_1\)};
            \node (c_2) at (1, -13) {\(c_2\)};
            \node (c_m) at (1, -16) {\(c_m\)};
            
            \node (t) at (-9,-23) {t}; 

            %vertical dots
            \node[draw = white, fill = white] () at (-15, -14) {$.$};
            \node[draw = white, fill = white] () at (-15, -14.5) {.};
            \node[draw = white, fill = white] () at (-15, -15) {.};
            \node[draw = white, fill = white] () at (-15, -15.5) {.};
            \node[draw = white, fill = white] () at (-15, -16) {.};

            \node[draw = white, fill = white] () at (-2.5, -14) {.};
            \node[draw = white, fill = white] () at (-2.5, -14.5) {.};
            \node[draw = white, fill = white] () at (-2.5, -15) {.};
            \node[draw = white, fill = white] () at (-2.5, -15.5) {.};
            \node[draw = white, fill = white] () at (-2.5, -16) {.};

            \node[draw = white, fill = white] () at (1, -15) {.};
            \node[draw = white, fill = white] () at (1, -14.5) {.};
            \node[draw = white, fill = white] () at (1, -14) {.};

            \path(v_{1,0})edge[black, bend left = 50, ->](v_{1,1});
            \path(v_{1,0})edge[black, bend right = 50, <-](v_{1,1});
            \path(v_{1,1})edge[black, bend left = 50, ->](v_{1,2});
            \path(v_{1,1})edge[black, bend right = 50, <-](v_{1,2});
            \path(v1)edge[black, bend left = 50, ->](v_{1,2m+1});
            \path(v1)edge[black, bend right = 50, <-](v_{1,2m+1});
    
            \path(v_{2,0})edge[black, bend left = 50, ->](v_{2,1});
            \path(v_{2,0})edge[black, bend right = 50, <-](v_{2,1});
            \path(v_{2,1})edge[black, bend left = 50, ->](v_{2,2});
            \path(v_{2,1})edge[black, bend right = 50, <-](v_{2,2});
            \path(v2)edge[black, bend left = 50, ->](v_{2,2m+1});
            \path(v2)edge[black, bend right = 50, <-](v_{2,2m+1});
            
            \path(v_{1,0})edge[black, bend right, ->](v_{2,0});
            \path(v_{2,0})edge[black, bend right, ->](v_{3,0});
            \path(s)edge[black, bend right, ->](v_{1,0});
            \path(s)edge[black, bend left, ->](v_{1,2m+1});
            \path(v_{1,2m+1})edge[black, bend left, ->](v_{2,2m+1});
            \path(v_{2,2m+1})edge[black, bend left, ->](v_{3,2m+1});
            \path(v_{1,2m+1})edge[black, bend right = 5, ->](v_{2,0});
            \path(v_{1,0})edge[black, bend left = 5, ->](v_{2,2m+1});
            \path(v_{2,2m+1})edge[black, bend left = 20, ->](v_{3,0});
            \path(v_{2,0})edge[black, bend right = 20, ->](v_{3,2m+1});

            \path(c_m)edge[black, bend left = 10, ->](v_{i,2m-1});
            \path(v_{i,2m})edge[black, bend left = 10, ->](c_m);
            
            \path(c_m)edge[black, bend left = 20, ->](v_{j,2m-1});
            \path(v_{j,2m})edge[black, bend left = 10, ->](c_m);
            
            \path(c_m)edge[black, bend right = 10, ->](v_{k,2m-1});
            \path(v_{k,2m})edge[black, bend right = 10, ->](c_m);

            \path(v_{n,0})edge[black, bend right, ->](t);
            \path(v_{n,2m+1})edge[black, bend left,  ->](t);
            
        \end{scope}
        \end{tikzpicture}
        \end{center}
        
        \begin{center}
        \textit{мал. 2 граф \(G_{\phi}\) згенерований кон'юнктивною нормальною формою \(\phi = \bigwedge\limits_{i=1}^m \phi_{i}\)\\зведення задачі \tsat\hspace{0.05cm} до \dhampath}
        \end{center}
        \newpage
        

    \section{\normalfontРозв'язок}
    \qquad Немає загального розв’язку задачі негамільтонового циклу для довільного графу. Однак для деяких конкретних класів графів існують добре відомі алгоритми пошуку рішень.
    
        \subsection{\normalfont Планарні графи та теорема Грінберга}
            %\subsubsection{\normalfont Теорема Грінберга}
            \qquad В теорії графі, теорема Грінбенга є необхідною умовою для того, щоб планарний граф містив гамільтонів цикл, базуючись на довжинах граней циклів. Якщо граф не задовільняє цю умову, то він негамільтонів. Результат теореми широко використовувався для для доведедення того, що деякі планарні графи, побудовані з додатковими властивостями не є гамільтоновими. Наприклад, це може довести негамільтоновість деяких контрприкладів до гіпотези Тейта. Теорема названа вчесть латвійського математика Емануеля Грінбенга, який довів її в 1968 році.
            \\

            \textbf{Теорема Грінбенга:} Нехай \(G\) - скінченний планарний граф з гамільтоновим циклом \(C\), з фіксованим зображенням на площині. Тоді \(C\) \textit{розбиває} граф на зовнішню на внутрішню підмножини. Позначимо \(f_i\) і \(g_i\) - кількість \(i\)-кутних граней, які є зовнішніми або внутршінми відносно циклу \(C\). 
            
            Тоді виконується співвідношення: $$ \sum_{i\ge3} (i - 2)(f_i - g_i) = 0 $$

            Як наслідок цієї теореми, якщо вбудований планарний граф має лише одну грань, кількість сторін якої не рівна \( 2 \mod 3\), а всі решта граней мають кількість сторін, яка дорівнює \( 2 \mod 3\), то граф не є гамільтоновим. Щоб побачити це, розглянемо суму у формі, наведеній у формулюванні теореми, для довільного розбиття граней на дві підмножини, пораховані числами \(f_i\) і \(g_i\). Кожна грань, число сторін якої \(\ 2 \mod 3\), збільшує суму, на величину кратну трьом, через \(k - 2\) у члені, тоді як одна грань, що залишилася, не змінює величину суми. Отже, сума не кратна трьом і, зокрема, не дорівнює нулю. Оскільки немає способу розділити грані на дві підмножини, які дають суму згідно з теоремою Грінберга, не може бути гамільтонового циклу.

            Грінберг використовува свою теорему для знаходження негамільтонових циклів в кубічних поліедральних графах (планарний граф, в якому степінь кожної вершини дорівнює трьом) з високою циклічною зв'язністю ребер. Циклічна зв'язність ребер графа - це найменша кількість ребер, видалення яких залишає підграф з більш ніж однією циклічною компонентою. Граф Тутте з 46 вершинами та менші кубічні негамільтонові поліедральні графи, отримані з нього, мають циклічну реберну зв’язність рівну трьом. Грінберг використовував свою теорему, щоб знайти негамільтонів кубічний многогранний граф із 44 вершинами, 24 гранями та циклічною зв’язністю ребер рівній чотирьом, а також інший приклад із 46 вершинами, 25 гранями та циклічною зв’язністю ребер п’ять, максимальна можлива циклічна зв’язність ребер для кубічного плоского графа, відмінного від \(K_4\). У наведеному прикладі всі обмежені грані мають або п’ять, або вісім ребер, обидва з яких є числами \( 2 \mod 3\), але необмежена грань має дев’ять ребер, що не дорівнює \( 2 \mod 3\). Отже, згідно з теоремою Грінберга , граф не може бути гамільтоновим.

            Теорема Грінберга також використовувалась для пошуку планарних гіпогамільтонових графів (граф який є негамільтоновим, але вилучення довільної вершини робить його гамільтоновим). Конструкція знову змушує всі грані крім однієї мати кількість ребер, рівну \( 2 \mod 3\). 

            Також, теорему можна застосовувати для аналізу гамільтонових циклів в окремих планарних графах, таких як узагальнений граф Петерсона, знаходячи великі планарні підграфи і доводити їх негамільтоновість за допомогою вище зазначеної теореми і роблячи висновок, що довільний гамільтоновий цикл повинен включати деякі ребра, що не входять у ці підграфи.

            Слід зазначити, що критерій Грінберга застосовний до деяких непланарних графів. Робертсон та Бонді довели, що узагальнений граф Петерсона \(P(n,2)\) - негамільтоновий, якщо \( n \equiv 5 \pmod{6}\). Також, Томасон довів, що існує рівно 3 гамільтонових цикла, якщо \( n \equiv 3 \pmod 6\). Гамільтонові цикли у всіх інших графах Петерсона було пронумеровані Швенком.

            Існують планарні негамільтонові графи у яких усі грані мають 5 або 8 ребер. Для таких графів, формула Грінберга, взята за модулем три, завжди задовольняється будь-яким розбиттям граней на дві підмножини, що запобігає застосуванню його теореми для доказу негамільтонності в цьому випадку.

            Неможливо використати теорему Грінберга для пошуку контрприкладів до гіпотези Барнетта про те, що кожен кубічний дводольний поліедральний граф є гамільтоновим. Кожен кубічний дводольний поліедральний граф має розбиття граней на дві підмножини, що задовольняють теоремі Грінберга, незалежно від того, чи має він також гамільтонів цикл.
            
            \subsection{\normalfont Алгоритмічні розв'язки}
            \subsubsection{\normalfont Алгоритм Мартелло, алгоритм Френка Рубіна}
                
                \qquadРаннім точним алгоритмом для знаходження гамільтонового циклу на орієнтованому графі був перечислювальний алгоритм Мартелло. 
            
                Пошуковий алгоритм Френка Рубіна ділить ребра графа на три класи: ті, які мають бути на шляху, ті, які не можуть бути на шляху, і невизначені. Під час пошуку набір правил прийняття рішень класифікує невизначені ребра та визначає, зупинити чи продовжити пошук. Алгоритм ділить граф на компоненти, які можна розв’язувати окремо.
            \subsubsection{\normalfont Алгоритми Белмана, Хелда та Карпа}

            \qquad Зазначені алгоритми, що базуються на динамічному програмуванні, можуть розв'язувати задачу за час \(\mathcal{O}(n^2 2^n)\). У цьому методі для кожного набору \(S\) вершин і кожної вершини \(v\) в \(S\) визначається, чи існує шлях, який точно охоплює вершини в \(S\) і закінчується в \(v\). Для кожного вибору \(S\) і \(v\) існує шлях для \((S, v)\) тоді і тільки тоді, коли \(v\) має сусіда \(w\) такого, що існує шлях для \((S - v, w)\), який можна знайти з уже обчисленої інформації в динамічній програмі.

            \subsubsection{\normalfont Алгоритм Андреаса Бйорклунда}
                \qquad Андреас Бйорклунд запропонував альтернативний підхід, використовуючи принцип включення-виключення, щоб зменшити задачу підрахунку кількості гамільтонових циклів до простішого підрахунку, підрахунку покриттів циклу, яку можна вирішити шляхом обчислення певних детермінантів матриці. Використовуючи цей метод, він показав, як вирішити проблему гамільтонового циклу в довільних графах з \(n\) вершинами за допомогою алгоритму Монте-Карло за час \(\mathcal{O}(1,657n)\); для дводольних графів цей алгоритм можна додатково вдосконалити до часу \(\mathcal{O}(1,415n)\).
            \subsection{\normalfont Складність}

            Задача є \(coNP\)\textit{- повною} для таких класів графів:
            \begin{itemize}
                \item дводольні графи
                \item неорієнтовані планарні графи максимального степеня три
                \item орієнтовані планарні графи з прямим і зовнішнім ступенем не більше двох
                \item неорієнтовані планарні 3-регулярні дводольні графи без мостів 
                \item 3-зв'язні 3-регулярні дводольні графи
                \item підграфи графа квадратної сітки
                \item кубічні підграфи графа квадратної сітки
            \end{itemize}
            \\
        Але все ж існують класи, для яких задачу можна вирішити за поліноміальний час:
            \begin{itemize}
                \item 4-зв’язні планарні графи завжди є гамільтоновими за результатом Тутте, і обчислювальну задачу пошуку гамільтонового циклу в цих графах можна виконати за лінійний час шляхом обчислення так званого шляху Тутте.
                \item Тутте довів цей результат, показавши, що кожен 2-зв’язний плоский граф містить шлях Тутте. Шляхи Тутте, у свою чергу, можуть бути обчислені за квадратичний час навіть для 2-зв’язних планарних графів, які можуть бути використані для знаходження гамільтонових циклів і довгих циклів у узагальненнях плоских графів.
            \end{itemize}
    \newpage
    \section{\normalfontСписок використаної літератури}
    
    \begin{thebibliography}{9}
        \item Adrian She, \emph{Hamiltonian Path is NP-Complete}, 2020.
        \item Dan Archdeacon, \emph{An Historical Account of the Non-Hamiltonian Cycle Problem}.
        \item Heping Jiang, \emph{Non-Hamilton cycle sets of having solutions and their properties}, 2019.
        \item Zaks, Joseph, \emph{Non-Hamiltonian non-Grinbergian graphs}, 1977.
        \item Emanuels Grinbergs, \emph{On planar regular graphs degree three without Hamiltonian cycles}, 2009.
        \item G.N. Robertson, \emph{Graphs under Girth, Valency, and Connectivity Constraints}, Ph. D. Thesis, University of Waterloo, Ontario, Canada, 1968.
        \item J.A. Bondy, \emph{Variations on the hamiltonian theme, Canad. Math. Bull}, 1972 57-62.
        \item A.J. Schwenk, \emph{Enumeration of Hamiltonian cycles in certain generalized Petersen graphs}, 1989.
        \item Chiba, Norishige, Nishizeki, Takao, \emph{The Hamiltonian cycle problem is linear-time solvable for 4-connected planar graphs}, 1989.
        \item Schmid, Andreas; Schmidt, Jens M., \emph{Computing Tutte Paths}, 2018.
        \item Björklund, Andreas, \emph{Determinant sums for undirected Hamiltonicity}, 2010.

    \end{thebibliography}


    
	\end{spacing}
 
\end{document}
